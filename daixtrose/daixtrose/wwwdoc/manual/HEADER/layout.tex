\pagestyle{headings}

\setlength{\parindent}{0pt}
\setlength{\parskip}{5pt plus 5pt minus 0pt}
%\setlength{\parskip}{5pt plus 5pt minus 2pt}

\frenchspacing
\sloppy

\setlength{\voffset}{-5mm}
\setlength{\hoffset}{0mm}
\setlength{\headsep}{15mm}
\setlength{\textwidth}{159mm}
\setlength{\textheight}{231mm}
\setlength{\topmargin}{0pt}
\setlength{\oddsidemargin}{0pt}
\setlength{\evensidemargin}{0pt}
\setlength{\footskip}{1cm}
\setlength{\marginparwidth}{60pt}

% --------------------------------------------


\setlength{\abovedisplayskip}{3pt plus 3pt minus 0pt}
\setlength{\belowdisplayskip}{3pt plus 3pt minus 0pt}
\setlength{\abovedisplayshortskip}{3pt plus 3pt minus 0pt}
\setlength{\belowdisplayshortskip}{3pt plus 3pt minus 0pt}

%\numberwithin{equation}{section}

% multline indents like normal equations!
\makeatletter
\setlength{\multlinegap}{\@mathmargin}
\makeatother

\renewcommand{\baselinestretch}{1.2}
%===============================================================================
% private Laengen

\newlength{\imagewidth}
\setlength{\imagewidth}{149mm}
\newlength{\halfimagewidth}
\setlength{\halfimagewidth}{74mm}

% So kontrolliert man Laengen:
% \message{The length is: \the\abovedisplayskip}
% \the\textwidth
% \showthe\textwidth


%===============================================================================
% Fussnotenbehaviour
%-------------------------------------------------------------------------------
\makeatletter
%
\renewcommand\footnoterule{%
  \kern-3\p@
  \hrule\@width1\columnwidth
  \kern2.6\p@}
\@addtoreset{footnote}{chapter}
%
\renewcommand{\@makefnmark}{\mbox{$^{\@thefnmark}$}}
%
\renewcommand{\@makefntext}[1]%
   {\noindent{\@makefnmark}#1}
%
\makeatother
%===============================================================================



%===============================================================================
% FONT! nochmal endgueltig verdrahten
%-------------------------------------------------------------------------------
% Abk�rzungen:
% pag - AvantGarde - schrift
% pcr - echtes Courier
% phv - Original Helvetica
% ppl - ein ``calligraphiertes'' Times -> Buch?
% pbk - �hnlich ppl, aber breiter
% put - h��lich, �hnlich times
% pnc - �hnlich Times, etwas verschn�rkelter
% pzc - NICHT EX.
% ptm - Das schoenere times aus dem times-package
%\renewcommand\sfdefault{phv}%               use helvetica for sans serif
\renewcommand\familydefault{\sfdefault}%    use sans serif by default

%===============================================================================


%===============================================================================
% An- und Abschalten von "unwichtigen" Umformungen ueber ein Flag
%-------------------------------------------------------------------------------
\newcommand{\textselection}{all}

\ifthenelse{\equal{\textselection}{nearly all}}{
% long Version, without the complements
\includeversion{DetailOfAlgebra}
\excludeversion{ReplacementForDetailOfAlgebra}
\includeversion{UnimportantAlgebra}
}{
% else: short version
\excludeversion{DetailOfAlgebra}
\includeversion{ReplacementForDetailOfAlgebra}
\excludeversion{UnimportantAlgebra}
}

% really everything that was typed
\ifthenelse{\equal{\textselection}{all}}{
\includeversion{DetailOfAlgebra}
\includeversion{ReplacementForDetailOfAlgebra}
\includeversion{UnimportantAlgebra}
}{}
%===============================================================================


%%% Local Variables: 
%%% mode: latex
%%% TeX-master: "../MAIN/main"
%%% End: 
