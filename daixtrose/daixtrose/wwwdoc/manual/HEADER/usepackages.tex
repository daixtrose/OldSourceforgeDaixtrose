\usepackage{makeidx}

% ???
\usepackage{latexsym}



%===============================================================================
% AMS
%-------------------------------------------------------------------------------
\usepackage[intlimits,fleqn,tbtags]{amsmath}
%\usepackage[leqno]{amsmath}
\usepackage{amssymb}
%\usepackage{amsthm}

\usepackage{amsthm}
%===============================================================================


%\usepackage[german]{babel}

\usepackage{epsfig}
\usepackage{float}
%\usepackage{floatfig} gibt es nicht

%\usepackage{moreverb}
%\usepackage{verbatim}
\usepackage{fancyvrb}

%\usepackage{verbatimfiles}

\usepackage[obeyspaces]{url}
\usepackage{hyperref}
%\usepackage[obeyspaces]{somecode}


\usepackage{wrapfig}
\usepackage{floatflt}
\usepackage{color}
\usepackage{calc}
\usepackage{ifthen}
\usepackage{version}



% FIXIT: ????
%\usepackage{oldgerm}

%===============================================================================
% pstcol.sty ist die Anpassung des color-Paketes an die Interna von PSTricks
%-------------------------------------------------------------------------------
%\usepackage{pstcol,pstricks}
%===============================================================================


%Korrekte Leerzeichen nach Befehlen 
\usepackage{xspace}

%===============================================================================
% Die leidige Font-Geschichte
%-------------------------------------------------------------------------------
%\usepackage{avant}
%\usepackage{bookman}
%\usepackage{times}

\usepackage{amsfonts}
%===============================================================================


%===============================================================================
% Umlaute richtig ...
%-------------------------------------------------------------------------------
\usepackage[isolatin]{inputenc}
%===============================================================================

%\usepackage{multicol}


%===============================================================================
% In grau am unteren Rand das Datum ...
%-------------------------------------------------------------------------------
\newcommand{\MyDate}{\number\day. \ifcase\month\or Januar\or
  Februar\or Maerz\or April\or Mai\or Juni\or Juli\or August\or September\or
  Oktober\or November\or Dezember\fi \space\number\year}

\usepackage[none,bottom]{draftcopy}
\draftcopyName{-------------------- Entwurf \MyDate%
 -------------------------}{215}
%===============================================================================



%===============================================================================

% does not work well
%\usepackage{listings}

%%% Local Variables: 
%%% mode: latex
%%% TeX-master: "../MAIN/main"
%%% End: 
