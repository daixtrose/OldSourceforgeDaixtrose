\section{Introduction}
\label{sec: Introduction}

  \hfill
  \begin{minipage}[t]{0.4\textwidth}
    [...] because when You are [...] of very little brain, 
    and You think of things, you find sometimes that a thing 
    which seemed very thingish inside you is quite different 
    when it gets out into the open and has other
    people looking at it." 
    \hfill (A.A. Milne)
  \end{minipage}


\vspace{5em}

Publishing code is always a risky thing to do. People look at it and immediately
find the flaws and the bugs and the places where the author's thoughts have gone off track.

Publishing code also is a chance. A chance for you and for me. For you it is the
chance to not having to go through the brain-damage that is required for this
kind of library, but simply use it and enjoy it.  In addition to that you can also
get some ideas while looking at the code and create something that is
better than this here.  For me it is the chance to obtain feedback and
correction. Both are things I am in need of all days. I~think that this way is
the best one to improve the quality of the library (and maybe my programming skills).

Therefore do not hesitate to give me some feedback. The \Daixtrose homepage mentions
a couple of mailing list. Use them. I read those messages as often as possible.

And now: Enjoy!



%%% Local Variables: 
%%% mode: latex
%%% TeX-master: "../MAIN/main"
%%% End: 
