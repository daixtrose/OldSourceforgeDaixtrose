

\section{Installation Instructions}
\label{sec: Installation Instructions}


\subsection{System Requirements}
\label{sec: System Requirements}

\Daixtrose is (or at least should be) platform independent. It's only a \CC
header library, nothing more.
If You want to use the GNU autotools based installation procedure which
compiles the testsuite, any unix system or a similar environment like cygwin is
nice to have, but this library is intended to be platform independent.

\subsection{Software Requirements}
\label{sec: Software Requirements}

\Daixtrose depends on the following software
%

\begin{itemize}
\setlength{\itemsep}{5pt minus 5pt}
\setlength{\parsep}{0pt}
\setlength{\parskip}{0pt}


\item A modern high performance ANSI compliant \CC compiler.  
  
  This library exploits most of the bright and dark corners of \CC templates.
  Therefore a rather standard conformant C++ compiler is needed.  This library
  was tested with Intel's \CC version 8.2 and 9.0 and should compile with
  gcc-3.3.4 or higher. Micr*s*ft's compilers VC7 and VC8 are not supported yet
  (no project files provided), but are expected to compile the code without
  problems.


%\item the \somecode{boost} library (see \url{http://www.boost.org})
\item the \somecode{boost} 
%\href{http://www.boost.org}{boost} 
library (see \url{http://www.boost.org})
%\item the \somecode{loki} library (see \url{http://sourceforge.net/projects/loki-lib})
\end{itemize}



This library is completely header-based and a simple 
\somecode{\#include "SomeHeader.h"} will do. The most important one is
\somecode{daixtrose/Daixt.h} which includes all of the core engine. But of
course You may \somecode{\#include} the files in a more refined way.

\Daixtrose is shipped with a fixed-sized matrix and vector package.
\somecode{\#include "tiny/TinyMatAndVec.h"} in order to use it.
\Daixtrose also comes with a (yet incomplete) linear algebra package (sparse
matrix + vector). In order to use these one needs to 
\somecode{\#include "linalg/Linalg.h"}

\subsection{Autotools Based Installation Procedure}
\label{sec: Autotools Based Installation Procedure}

To ease the compilation of the testsuite and in order to support possible future
diversification of the library, daixtrose is shipped with a GNU autotool-based
configure shell script.

Perform the following steps (probably with some modifications), if You like to
get the examples built (it is recommened to create a new directory
\emph{outside} of this directory, to keep the original source tree clean):

\begin{code}
  $ cd ..  
  $ mkdir build-daixtrose 
  $ cd build-daixtrose 
  $ ../daixtrose-0.0.1/configure \ 
  CXX=icpc CPPFLAGS="-g -wd76 -ansi -O3 -Ob2" \ 
  --prefix=$HOME/daixtrose-0.0.1 \ 
  --with-boost=/path/to/boost/prefix
  
  $ make 
  $ make check 
  $ make install
\end{code}

At present the command \somecode{make} will do nothing.

The command \somecode{make check} is optional and may be omitted. 
It evokes compilation of all the examples in the directory \somecode{demos}
and can be regarded as a compiler test.

The command \somecode{make install} will install the headers under 
\somecode{prefix/include}.%$ 

For further control over the installation procedure consider the
output of \somecode{../daixtrose-0.0.1/configure --help}.

%\$

\subsection{Update of the Library}
\label{sec: Update of the Library}

This library undergoes permanent evolution. So it is a good idea to consider regular
updates. Visit the homepage of \Daixtrose at \somecode{http://daixtrose.sf.net}
for download instructions. 

A more frequent update is possible through checkouts of the cvs repository.  The
following commands are sufficient to check out the latest version:

\begin{code}
  $ cvs -d:pserver:anonymous@cvs.sourceforge.net:/cvsroot/daixtrose login 
  $ cvs -z3 -d:pserver:anonymous@cvs.sourceforge.net:/cvsroot/daixtrose \
    co daixtrose 
\end{code}


You may also visit the cvs web-interface at
\url{http://sourceforge.net/cvs/?group_id=75079}

%\begin{code}
%http://sourceforge.net/cvs/?group_id=75079
%\end{code}




%%% Local Variables: 
%%% mode: latex
%%% TeX-master: "../MAIN/main"
%%% End: 
