\section{Quick Tour}
\label{sec: Quick Tour}

You want to know what \Daixtrose is good for? This is an attempt to explain it
within 10 minutes. Built on top of the core \Daixtrose library are two matrix and
vector library snippets, one for applications where the sizes of matrices or
vectors are known at compile time and one for large scale applications with 
sparse storage of the matrix and a more flexible approach.

Both primarily were intended as demo code to show the power of \Daixtrose, but
they already do a decent job in the authors doctoral thesis project, a
time-implicit finite element solver.  Therefore you can profit from Daixtrose
even if you do not want to build your own Expression Template library on top of
\Daixtrose, but rather are only in need of a fairly good matrix and vector
library.

All code examples presented here can be found in the directory
\somecode{src/demos/quicktour/} and will get compiled during a \somecode{make
  check} as described in the installation instructions (\ref{sec: Installation
  Instructions}).

% ==============================================================================
% //////////////////////////////////////////////////////////////////////////////

\subsection{Fixed-Sized Matrix and Vector}
\label{sec: Fixed-Sized Matrix and Vector}

\somecode{TinyMatrix} and \somecode{TinyVector} are inspired by Todd
Veldhuizen's \somecode{blitz++} library. They are one-based (first index is 1
not 0) and can be used like this:

\inputcppfile{../../../src/demos/quicktour/TinyMatrixAndVector.C}
%\VerbatimInput[xleftmargin=10mm, fontseries=b, fontsize=\small,numbers=none]{../../../src/demos/quicktour/TinyMatrixAndVector.C}

% ==============================================================================
% //////////////////////////////////////////////////////////////////////////////

\subsection{Linalg: Sparse Matrix and Vector}
\label{sec: Linalg: Sparse Matrix and Vector}

The sparse matrix implementation presented here assumes a row-oriented usage
like e.g. in matrix-vector multiplications or additions resp. subtractions. The
matrix data is stored in a \somecode{std::vector} of user-definable
\somecode{RowStorage}s that must behave like a \somecode{std::map<size\_t, T>},
which is the default. Each line in the matrix is represented by a
\somecode{RowStorage}, where the keys store the colum number.  In addition to
the common matrix interface (index operator, etc.) the user additionally has
read-only access to each line or column, which allows individual manipulations
if needed.  Like for \somecode{TinyMatrix} the sparse matrix and the compatible
vector are one-based.

Due to the sparse representation of the matrix it is not completely possible to
avoid all temporaries during assignment of matrix expressions, but the run-time
and memory overhead is negligible for matrices with more than 100 lines. The
assignment loop creates new rows and inserts them at the correct places of the
asignee. 

Self assignment and assignment from expressions which contain the
assigned Matrix or Vector will be detected at runtime and will safely produce a
temporary before final assignment. 

To allow complete, fine grained control over memory management, the allocators
for the \somecode{RowStorage} can be optionally chosen to be different from
\somecode{std::allocator}

Here is some code which demonstrates the features which are available.
(Mentioning the allocators in the typedefs is not needed, but for demonstration
purposes the defaults are shown in the following example).

\inputcppfile{../../../src/demos/quicktour/Linalg.C}


% ==============================================================================
% //////////////////////////////////////////////////////////////////////////////

\subsection{A Sparse Matrix of Fixed-Sized Matrices}
\label{A Sparse Matrix of Fixed-Sized Matrices}

Of course you can have a sparse matrix of fixed-sized matrices. That is easy
enough to achieve: 

\begin{code}
  typedef TinyMat::TinyQuadraticMatrix<double, 3> BlockMatrix;
  typedef Linalg::Matrix<BlockMatrix> Matrix;

  typedef TinyVec::TinyVector<double, 3> BlockVector;
  typedef Linalg::Vector<BlockVector> Vector;
\end{code}

You can find complete examples in 
\somecode{src/demos/linalg/TestBlockedMatAndVec.C}
\somecode{src/demos/linalg/TestPrintingOfBlockedMatrix.C}


% ==============================================================================
% //////////////////////////////////////////////////////////////////////////////

\subsection{Using the Core Engine (Building Expression Templates at Home)}
\label{sec: Using the Core Engine (Building Expression Templates at Home)}


\subsubsection{Delaying Evaluation}


\Daixtrose was designed to provide a convenient way to apply Expression Template
techniques to any kind of \CC classes without the need to modify any line of
code of these classes. As an example consider you have written a very elaborated
high performance \somecode{class X}. The only things missing are the operators
and other code which would allow objects of type \somecode{X} to be used within
expressions. No problem for us, we have \Daixtrose:
%
\inputcppfile{../../../src/demos/quicktour/Mini.C}

Congratulations! You have combined the power of \somecode{class X} with the
invaluable features of \Daixtrose. This code already compiles.  Of course there
is some lack of functionality, but we fix that in a minute.

What is important here is that with the help of \somecode{namespace
  Daixt::DefaultOps} any expression gets mapped unto an Expression Template
tree.  E.g. the expression above results in contructor calls for
\somecode{Expr<T>}, \somecode{BinOp<...>} and \somecode{UnOp<...>} and yields an
object of type\footnote{This expression tree has been simplified. All class
  names reside inside namespaces, which were omitted for the sake of readability
%Daixt::Expr<Daixt::BinOp<Daixt::BinOp<Daixt::BinOp<Daixt::BinOp<X, X, Daixt::DefaultOps::BinaryPlus>, Daixt::UnOp<X, Daixt::DefaultOps::RationalPower<4, 3>>, Daixt::DefaultOps::BinaryMultiply>, Daixt::UnOp<X, Daixt::DefaultOps::RationalPower<2, 1>>, Daixt::DefaultOps::BinaryDivide>, Daixt::BinOp<Daixt::UnOp<X, Daixt::DefaultOps::UnaryMinus>, X, Daixt::DefaultOps::BinaryMultiply>, Daixt::DefaultOps::BinaryDivide>>
} (big breath)
%
\begin{code}
Expr<
     BinOp<
           BinOp<
                 BinOp<
                       BinOp<X, X, BinaryPlus>, 
                       UnOp<X, RationalPower<4, 3> >,
                       BinaryMultiply
                       >, 
                 UnOp<
                      X, 
                      RationalPower<2, 1> 
                      >, 
                 BinaryDivide
                 >, 
           BinOp<
                 UnOp<X, UnaryMinus>, 
                 X, 
                 BinaryMultiply
                 >, 
           BinaryDivide
           >
     >
\end{code}

This is what \Daixtrose offers. Users (or library authors) do not have to retake
the tedious task of reinventing yet another Expression Template machinery but
can rely on the \Daixtrose Expression Template core engine.

Expression Templates avoid temporaries through delayed evaluation.  The first
step of a delayed evaluation, namely the delaying, has already been taken in the
code above. Now it is up to the user to perform the evaluation. 

\subsubsection{Expression Evaluation}

Evaluation can
be achieved with a set of template functions (or a functor with an overload set
of template member functions alias \somecode{operator()}.  These functions can
rely on the public interfaces of the \Daixtrose classes, especially on the
member functions  
%
\begin{code}
Expr<T>::content()
BinOp<...>::lhs()
BinOp<...>::rhs()
UnOp<...>::arg()
\end{code}

To illustrate this, let us extend the example above:
%
\inputcppfile{../../../src/demos/quicktour/Mini.2.C}


\subsubsection{Assignment from an Expression}

In a similar manner assignment can be achieved by reusing existent public
interfaces. Consider a case where \somecode{class X} has a public member
function that allows resetting some internal value. Here we show an example
where no temporary object of type \somecode{X} is created, but only temporray
objects of the contained data type are created.  Applying this technique to
classes which contain large data sets like e.g. sparse matrices is the key to
runtime efficiency. 

 

\inputcppfile{../../../src/demos/quicktour/Mini.3.C}

Note that in the example shown here assignment from an expression can be
performed without modifying \somecode{class X}, but of course adding an
appropriate \somecode{operator=} to \somecode{class X} is the way of choice in
cases where \somecode{class X} was designed to rely on \Daixtrose:

\begin{code}
class X 
{
  std::string Name_;

public:
  X(const std::string& Name) : Name_(Name) {}

  template<class T>
  X& operator=(const Daixt::Expr<T>& E)
  {
    Name_ = Evaluate(E));
    return *this;
  }

  std::string Name() const { return Name_; }
};
\end{code}

In order to ease the output of expressions,  
\somecode{namespace Daixt::OutputUtilities} contains everything needed to
output any expression. This feature works for any leaf \somecode{class T} that has
a \somecode{std::ostream\& operator<<(std::ostream\& os, const T\& t)} defined.
See how this feature was used in the sparse matrix example 
\ref{sec: Linalg: Sparse Matrix and Vector} above.  
 

\subsubsection{Disambiguation}

Sometimes it is a good idea to let the compiler emit an error 
if incompatible types are combined in an exprression. Perhaps one 
does not want a 
\somecode{Vector} added to a \somecode{TesorRankOne} to pass the compiler
ending in long debugging sessions. 

\Daixtrose takes care of preventing the combination of incompatible types into 
one expression and provides fine-grained control over its behaviour.
The disambiguation mechanism compares types. Per default it takes the type of
the leaves of the expression tree (\somecode{class X} in our example) as
disambigution type. 

There are two other possibilities for providing a disambiguation type:
classes can contain a typedef like in
%
\begin{code}
struct X_Disambiguation {};

class X {
  typedef X_Disambiguation Disambiguation;
};
\end{code}
%
or one can specialize \somecode{Daixt::Disambiguator} which again allows
plugging class X into the Expression Template engine without modification:
%
\begin{code}
class X {};

struct X_Disambiguation {};

namespace Daixt 
{ 
template <> 
struct Disambiguator<X> 
{
  static const bool is_specialized = true;
  typedef X_Disambiguation Disambiguation;
};
}
\end{code}

This techique allows one to combine classes 
which differ in type, but where it makes perfect sense to combine them in
expressions without error:

\inputcppfile{../../../src/demos/quicktour/Mini.4.C}

Performing an operation on some object may result in another type. Consider the
multiplication of a matrix with a vector. The result is a vector expression.
Daixtrose allows to communicate this fact via specializations of
BinOpResultDisambiguator and UnOpResultDisambiguator:

\inputcppfile{../../../src/demos/quicktour/Mini.5.C}

\subsubsection{Value Semantics vs. Reference Semantics}

\Daixtrose default behaviour is to use value semantics while building the object
that represents the expression tree. This means that all leaves (operands) get
copied once. Users can modify this behaviour by specializing
template \somecode{class CRefOrVal}:

Adding the following code to the example above prevents matrices and vectors to
get copied.  
\begin{code}
namespace Daixt 
{

template <class T>
struct CRefOrVal<MyOwn::Matrix<T> > 
{
  typedef Daixt::ConstRef<MyOwn::Matrix<T> > Type;
};

template <class T>
struct CRefOrVal<MyOwn::Vector<T> > 
{
  typedef Daixt::ConstRef<MyOwn::Vector<T> > Type;
};

}
\end{code}

\Daixtrose now uses const references to the leaf objects. Of course
the user then is responsible for avoiding dangling references by ensuring
the lifetime of all leaves is longer than the lifetime of the expression.


\subsubsection{Adding Polymorphic Behaviour to Expression Templates}

It's a kind of magic, but it works well.
\somecode{Daixt::Expr<T>} inherits from \somecode{FeaturesOfExpressions} using
the Couriously Recursive Template Pattern (due to Geoffrey Furnish, also called
the Couriously Recurring Pattern, so let's call it CRTP). Users can specialize
this class which makes it possible to add publicly accessible member functions to
expressions on the fly.

This feature is build on top of a disambiguation changing mechanism
which may help around some brain-damage in other situations, too.

But see yourself: a piece of code says more than further explanations:

\inputcppfilestrippedheader{../../../src/demos/ChangeDisambiguation/main.C}


\subsubsection{Compile Time Differentiation}

If you ever want to let the compiler analytically differentiate arbitrary
expressions, \Daixtrose is the library of choice. This is not exactly what the
compiler was invented for, but this feature is extremely useful e.g. for
creating generic versions of exact Newton solvers.

There is one restriction: compile time differentiation only works for
variables that differ in type. Of course one wants all variables to behave
similar, so the nearest solution is to use a template class that makes
the variables distinguishable \emph{at compile time} through the template
argument.

Some possible candidates are
\begin{code}
  template <std::size_t Number> class Variable;
\end{code}
%
or
\begin{code}
  template <const char Name[256]> class Variable;
\end{code}

The first version is used in \somecode{src/demos/simplify/TestSimplify.C}, the
second version in the following demo example. Differentiation was implemented
such that the product and sum rules are formally applied without any
simplification. This means that zero terms are not automatically removed.
Example: \somecode{Diff(a + b, b)} yields \somecode{0 + 1} or - in \CC terms - an
object of type \somecode{Expr<BinOp<...> >} instead of an
\somecode{Expr<IsOne<..> >}.

For production code the compile time zero-branch deletion is crucial for the
runtime performance. Therefore expression simplification is implemented
separately and can be applied to the result of a differentiation.  
Example: \somecode{Simplify(Diff(a + b, b))} yields the desired result \somecode{1}.

In order to speed up compilation for cases where these features are not needed
differentiation resides in \somecode{namespace Daixt::Differentiation} and expression
manipulation in \somecode{namespace Daixt::ExprManip}. 

The following example shows both features in action.

\inputcppfile{../../../src/demos/quicktour/Differentiation.C}


\subsubsection{Poor Man's Lambda Library}

Lambda expressions in C++ are one possible application of the Expression
Template technique. A stable, peer-reviewed and tested implementation can be
found within the \somecode{boost} library. If You feel like You still have to write
Your own lambda library (which is a true waste of time) then You can build it on
top of \Daixtrose. The following code snippet demonstrates how to communicate
Your ideas to the \Daixtrose library.

\inputcppfile{../../../src/demos/quicktour/PoorMansLambda.C}





% ==============================================================================
% //////////////////////////////////////////////////////////////////////////////


\subsection{Have a lot of fun}

This is the end of the Quick Tour.
Further Information can be gathered by looking at the files in directories
\somecode{demos}, \somecode{tiny} and \somecode{linalg}.

Please send me some feedback via the mail addresses at \url{http://daixtrose.sourceforge.net/contact.html}

Enjoy!



%%% Local Variables: 
%%% mode: latex
%%% TeX-master: "../MAIN/main"
%%% End: 
